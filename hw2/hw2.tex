\documentstyle[12pt]{article}

\pagestyle{empty}

\setlength{\textheight}{7in}
\addtolength{\topmargin}{-.5in}

\begin{document}                                                             

%\topmargin 0 in
%\headsep 0 in
%\addtolength{\textheight}{.2in}
\centerline{\large \sc Math 55} 

\smallskip

\begin{center}
{\large Homework 2\\
Due Thursday, 9/20/18}
\end{center}

\medskip

\medskip

\indent 1. How many ways can 12 distinct candy bars be
distributed to 4 distinct children such that child 1 gets 2 bars,
child 2 gets 2 bars, child 3 gets 4 bars, and child 4 gets 4
bars? \\
\newpage
\indent 2. Prove by induction:
$1+x+x^2+\cdots+x^n=\frac{1-x^{n+1}}{1-x}$, where $x\neq 1$ and
$n$ is a nonnegative integer. \\
\newpage
\indent 3. Prove by induction that for all integers $n\geq 0$, 
$$\int_0^{\infty} x^n e^{-x} dx = n!$$
\newpage
\indent 4.  Prove that
$\frac{1}{1\cdot2}+\frac{1}{2\cdot3}+\cdots+\frac{1}{2000\cdot2001}=
\frac{2000}{2001}$

a) using induction, and 

b) without using induction. (Hint: $\frac{1}{k(k+1)} = \frac{1}{k} - \frac{1}{k+1}$.) \\
\newpage
\indent 5. Prove by induction that for $n\geq 1$, $\sum_{k=0}^n k{n\choose k} = n2^{n-1}.$
(Hint: At some point (late) in the proof, you might want to replace $k$ with $(k-1) + 1$. \\
\newpage
\indent 6. Prove the identity above combinatorially. \\
\newpage
\indent 7. Suppose that $n\geq 3$ children are playing on a
playground, each holding a custard pie. Suddenly every child throws
their pie at the face of the child that is closest to them. Assume
that all the distances between the children are distinct so this rule
is well defined. Prove, by induction, that if $n$ is odd, then at
least one child has no pie thrown at her.

\end{document}
